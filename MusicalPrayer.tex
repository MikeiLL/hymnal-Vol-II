\documentclass[12pt]{article}
\usepackage{lingmacros}
\usepackage{tree-dvips}
\usepackage{setspace} % for \onehalfspacing and \singlespacing macros
\onehalfspacing

\usepackage{etoolbox}
\AtBeginEnvironment{quote}{\par\singlespacing\small}

\begin{document}

\section*{Musical Prayer}

Thank you all for  the opportunity to share and further refine my understanding of musical prayer. It's been a blessing to have been raised with a great interest in and reverence for the power of music to bring us closer to the divine and each other.

Wonderfully, this community welcomes atheists whom I hope will tolerate references to God in terms of perhaps, "the one who does not exist", "aspects of reality that defy rational interface" or some other fun applicable term.

There's a phrase in the Bhagavad Gita offering the idea that "Scriptures are of little value to [one] who sees God everywhere." For the rest of us, there are myriad systems, cosmologies, symbols, philosophies and other tools that can facilitate our journey through this mystery-filled fog of existence.

Iconoclastic occultist Aleister Crowley suggests that, "True religion is intoxication", and that "all acts which excite the divine in [us] are proper to the rite of invocation." Sufi tradition echoes this sentiment in spiritual metaphors of "wine" and "intoxication" as states of divine bliss and at the very least departure from the realm of rationality. We can achieve states of ecstatic spiritual bliss through music. In yogic terms this is an aspect of the bhakti tradition: the yoga of devotion. The soul's voicing of its aspiration. One of eight limbs of yoga, as described in the yoga sutras of Patanjali, the others which concentrate on ethical guidelines, physical postures, breath control, withdrawal of the senses, concentration, meditation and absorption.

Luminary twelfth century poet, composer and mystic Hildegarde von Bingen wrote, "Don't let yourself forget that God's grace rewards not only those who never slip, but also those who bend and fall. So sing! The song of rejoicing softens hard hearts. It makes tears of godly sorrow flow from them. Singing summons the Holy Spirit. Happy praises offered in simplicity and love lead the faithful to complete harmony, without discord. Don't stop singing."

\subsection*{Tantra and Prana}

Two concepts that come to us from India are tantra and prana. As it was explained to me during the Vedic period between about 1500 and 200 BCE the human body was viewed as separate from and perhaps even impeding our connection with the divine. Spiritual practitioners aimed to transcend the body and the material world. In tantric philosophy beginning in the Classical period between 200 BCE and 500 CE, the body could be considered a manifestation of and map toward the divine.

This ties in with the sanscrit word, prana, which signifies breath, life and cosmic energy. Prana yama practices are techniques for cultivating and conducting energy, perception and health.

One that I'll share quickly now is kapalabhati pranayama apparently means "the pranayama that makes the forehead and entire face lustrous." It alternates short, active exhalations with passive, reflexive inhalations. Ideally through the nose.

You can monitor the accuracy of your practice by placing a palm on your belly and holding a finger under your nose. Now inhale normally and engage, via the abdomenal muscles your diaghram to push out a short burst of air. You are looking for your belly to contract and finger to feel the air exiting the nostrils (or mouth). You'll also hear the sound of the little air burst on the exhalation. You may at first notice the belly coming out as opposed to in on exhalation, this is a common mistake, but generally is resolved with a little practice, perhaps even resolving some of the reverse-breathing that many of us do. Oxygenating the blood and brain stimulates the digestive organs, oxygenates the blood, reduces stress reaction and facilitates vocalization.

\subsection*{Presentation Outline}

Ideally this presentation will inspire us as a congregation to courageously express and explore more deeply the possibilities of our engagement with music. You've been very open, welcoming and supportive musical allies and I look forward to seeing how our collective musically prayerful endeavors evolve.

\subsection*{Loose Historical Overview}

Chanting as a spiritual practice is found in miriad cultures and religions with historical documentation going back at least as far as 1300 years before the birth of Christ. Various Old Testament books, especially the Psalms and the Chronicles, testify to the central function of music in temple worship. Psalms 150:4 which is an old testament book derived from the Torah, states, "Praise Him with the timbrel and dance; Praise Him with stringed instruments and flutes!"

Repetitive, sometimes call and response chanting is found is Byzanine, Gregorian, and other Christian traditions; Jewish cantillation, Hindu bhajan, Tibeten harmonic and Mongolian throat singing, Buddhist sutra, Sikh kirtan, the widely influential prayer calls of the Muslim muezzin; the Sufi zikr (which translates calls to remember); the orisha songs of the Nigerian Yoruba; Aboriginal Australian songlines; norito chants of the Shinto religion in Japan; Indigenous tribes of the Americas and countless other traditions utilize the practice as well as that of hymn singing.

Hymns are songs in praise of a one or more deities and are found in the earliest of religious texts. The collection of Vedic Sanskrit hymns, the Rigveda is one of the oldest known texts in the world. Hurrian Hymn to the Messopotanian goddess Nikkal is the oldest surviving substantially complete work of notated music in the world.

In ancient Egypt inscribed on the walls of the mid 14th century BCE tomb of Pharaoh Amenhotep IV was found the Great Hymn to the Aten:

\begin{quote}
How manifold it is, what thou hast made!
They are hidden from the face.
O sole god, like whom there is no other!
Thou didst create the world according to thy desire,
Whilst thou wert alone: All men, cattle, and wild beasts,
Whatever is on Earth, going upon feet,
And what is on high, flying with its wings
\end{quote}

A later excerpt continues:

\begin{quote}
You are in my heart,
\end{quote}

This hymn to sun god Aten has been compared to Psalm 104 in the Bible, which begins "barachi nafshi et adonai", "Bless the Lord, O my soul."

Hymnologist Arthur Weigall compared the two texts side by side, commenting that "In face of this remarkable similarity one can hardly doubt that there is a direct connection between the two compositions; and it becomes necessary to ask whether both Akhnaton's hymn and this Hebrew psalm were derived from a common Syrian source, or whether Psalm 104 is derived from this Pharaoh's original poem. Both views are admissible."

The Homeric Hymns to which references as far back as the 6th century BCE have been found are a collection of thirty-three anonymous Greek hymns celebrating gods from the Greek pantheon.

Early Christians are supposed to have sung hymns together as a passage from New Testament Colossians reads, "Let the message of Christ dwell among you richly as you teach and admonish one another with all wisdom through psalms, hymns, and songs from the Spirit, singing to God with gratitude in your hearts." Interestingly though in Roman Catholic churches by the outset of the Protestant Reformation, "priests chanted in Latin, and choirs of professional singers predominantly sang polyphonic choral music in Latin." Musician and author Paul S. Jones writes that at this time, "there was neither congregational song nor any church music in the common tongue." It was the work of Martin Luther and John Calvin during the Protestant Reformation that brought hymn singing back to the masses.

Initially hymnals didn't include music, only lyrics and meters by which familiar melodies could be applied. The congregation would learn the tune from the choir or a song leader. In around 1800 there was bitter debate over the "old way" of Old Testament style psalm chanting versus the "new way" of hymn singing.

Congregational music reading and literacy were the result of a broad social movement including regular educational hymnodic social events which were popular in part due to the limited social options at the time. A brilliant and prolific hymn composer named William Billings was considered the unrivaled champion of the "new way". He published his first book of original songs in 1770 and much of the text was drawn from Unitarian poets. Billings is credited with organizing the first church choir in America and by the late 1880s his music was known across the United States with nearly every songbook published in America including some of his work.

In a history of the music at First Parish Concord in Massachusetts, Eleanor Billings describes the era this way:

"(Billings) was part of a whole generation of singing masters who would travel from town to town holding classes on note-reading. These evenings were especially popular among the young people in the towns, as both young men and young women were welcome, and social events were few. Those who attended the classes came to be called the 'new way' singers, who read notes and soon graduated to 'anthem' music, rather than hymn or psalm singing, and before long were seated in their own section of the church, to lead the congregational singing in lieu of deacons, and even to sing some special pieces by themselves."

It was also around this time that shape notation–in which shapes were added to the noteheads in written music to help singers find pitches–were becoming a popular device in the documentation and dissemination of sacred and popular music in England and America, particularly in New England and the South.

Black American populations were largely illiterate during this period and cultivated a rich call and response approach to hymn singing. Organized in 1871 out of Nashville, Tennessee's Fisk University, the Fisk Jubilee Singers toured the United States and Europe in the 1870s and 1880s singing spirituals and hymns in the African American tradition and were credited with the popularization and preservation of this orally evolved music, a sibling of which is Gospel music which combines spirituals with elements of blues music.

We find evidence of all of these traditions in our own UU hymnals and musical activities.

\subsection*{Chakra System}

As our voices are an aspect of our physical manifestation, I want to work with one of the oldest and most popular interfaces between the physical and spiritual realms: the chakra system.

The word chakra translates as circle or wheel and the chakra system refers to energy centers. Many of us grew up pronouncing the word with a soft "ch", but apparently that would refer to some kind of pickle, so now we pronounce chakra like chocolate. Thank you Nancy LaNasa for clearing that up. There are varying but largely similar ideas about where the primary energy centers in the body are, whether they are called chakras or by other names. I'm told that the original system from Buddhist and Hindu tradition is a system that associates colors with the lower 5 of seven chakras, the colors being Gold at the root, Clear or White (representing the milk of Viṣṇu) in the naval area, Red at the solar plexus, Green at the heart and Blue or Black at the throat. The colors of the popular "new age" system correspond with those of the rainbow, which is less engaging but offers a familiar mapping by which to memorize and visualize these energy centers.

\begin{itemize}
  \item Red at the root, located at the base of the pelvis it the energy associated with our connection to the earth and material needs: food, digestion, shelter, safety.
  \item Orange in the naval region is the center of creativity and sexuality.
  \item Yellow at the solar plexus is the center of will.
  \item Green at the heart is where compassion and anger reside.
  \item Blue at the throat is the center of communication, both listening and speaking.
  \item Indigo at the third eye is the center of perception.
  \item Violet at the crown of the head is our connection with and to the divine.
\end{itemize}

The sanscrit names for these chakras are Mūlādhāra, Svādhiṣṭhāna, Maṇipūra, Anāhata, Viśuddha, Ājñā and Sahasrāra and it is common to practice meditations and visualizations involving one or more of these energy centers. The practices may include mudras or sacred ritual gesture associated with each chakra, as well as physical postures known as asanas and the use of bījākṣara or seed syllables.

In sanskrit bījākṣara and often simply referred to just by the term bīja meaning seed. Corresponding words in Japanese and Chinese are shuji and zhǒng zǐ. The bīja—a term common to both Hinduism and Buddhism—don't have a specific literal meaning, but are known to vibrate sympathetically with certain universal principles, or what Jung might have called archetypes.

The bīja for the root chakra is Lam, for the naval region Vam, Ram at the solar plexus, Yam the heart, at the throat Ham, the most well known bījākṣara Om resonating at the third (or as some refer to it first) eye and at the crown of the head the silent Om.

\subsection*{Lam meditation}

I'd like us to do a grounding meditation and chant together which combines the seed syllable Lam and its corresponding mudrā. The mudrā is called the prithvi mudrā or earth mudrā. To form this mudrā, bring the ring finger and thumb tips together, keeping the other three extended. The ring finger represents the earth element and the thumb represents the fire element. The joining of these two fingers is said to increase the earth element within the body. If you were seated near the ground you could touch the earth with the extended fingers. For now you may simply visualize this connection.

Take a full inhalation and repeat in the rhythm of your own full breaths, a low pitched "Lam" on the exhalation. Since we all have different lung capacity and vocal range, the pitches will vary and the syllables will overlap like waves on the chore.

Imagine roots of gold reaching down into the earth from the pelvic area and intertwining roots reaching up from the earth into your body. Be aware of the sounds of each others voices. Feel free to close your eyes for a few moments.

Our mother earth is the source and destination of our bodies. We are using the voice as a tool to deepen our connection with the earth along with our awareness of it and each other.

Now slowly return to the vibration of silence.

\subsection*{Nāda Yoga}

The Christian story of Genesis and many other creation stories including the concept of the big bang begin with a reference to sound. The ancient Egyptians believed that the god Thoth created the world through the power of sound. The Greek philosopher Pythagoras believed that the universe was created through the power of sound and that the planets and stars moved according to mathematical equations which corresponded to musical notes. The ancient Indian texts known as the Vedas describe the universe as being created through the power of sound.

The science of naad or nāda yoga is the yoga of sound and references two types of sound: anāhata (sharing its name with that of the heart chakra) is the sound that is not struck; the internal sound, and ahaṭa is the sound that is struck; external sound.

% In the ancient times before the Rig Veda (an ancient yogic text), people would sit in meditation and connect with cosmic consciousness. These people (Rishis) would see images and hear beautiful sounds. These sounds were called nāda. The Anahata nāda (inner sound) they experienced was recreated as Ahata nāda (external sound) and this is how Sanskrit became.

% Since the Sanskrit language came from Anahata nada, it is a language meant to be sung. https://www.brettlarkin.com/nada-yoga/

I've been told that the sound generated within your body would be overwhelmingly louder than any sound you could hear from the outside if it weren't filtered by the brain in a process neuroscience describes as auditory sensory gating. What we describe as "sound" is a combination of the vibration of air molecules and a process of filtering by the brain which science has only begun to understand.

The concept of utilizing the voice as a tool for manifestation and transformation is culturally ubiquitous. From Egypt to China, India to Africa and the Americas, the voice has been used to heal, to communicate with the divine, to create and to destroy. The phrase Abracadabra is said (though not without some debate) to be derived from the Aramaic phrase "I create as I speak." Hebrew Evra K'divra apparently translates, "I create as is the spoken word."

\subsection*{Kirtan}

Kirtan is a form of call and response chanting found in Hindi and Sikh traditions. The leader sings a line and the congregation repeats it. The approach to call and response chanting is also found in the Sufi tradition, in the music of the African diaspora, notably in the music of the Yoruba people of Nigeria and its Afro-Cuban and Afro-Brazilian descendants and as noted earlier in the gospel tradition. These sessions often last for many hours with various groups of musicians and singers taking turns in leadership and accompaniment.

In the eastern traditions, the chants may be or revolve around a mantra. The word mantra is derived from the sanscrit words manas, meaning mind, and tra, meaning tool. The word mantra is often translated as "mind tool" or "instrument of thought". The popular mantras are single or combinations of words so powerful that they have survived intact for generations, their potency growing with each new voice, stirring the content into the datasphere; into the collective unconscious; into our dna, each mantra an auditory glyph: a small package connecting and containing vast networks of information, stories, mythologies, phycological tools, scientific expression and magic.

In Ifa/Orisha tradition an example is Mojuba O Mojuba Orisha, meaning, "Orisha, I salute you." In the Hindu tradition, the Gayatri mantra is a popular mantra which is said to be the oldest known mantra in existence. Sarasvatī who shares a name with a legendary lost river, is the Hindu goddess of learning and through whom mantra came to humankind.

\subsection*{Om Asato Ma Sadgamaya}

The chant we're going to practice together as a call and response is said to be one of the oldest texts in the Upanishads. Upanishads I've been told translates as "to sit near." It's a collection of written documentations of ideas from the oral traditions of Hinduism. Each would be a long teaching going deeper and deeper toward a revealed reality which is the secret teaching the story is intended to reveal.

In the 1990s a few friends and I studied yoga with a man named Joe Brennan, who had lived for about seven years in the ashram of Swami Satchidananda. He would share this mantra at the close of each class. What we'll be manifesting here makes three requests: Lead me from non-being to being; Lead me from darkness to light; Lead me from death to eternal life. The chant begins with an Om and ends with Om peace peace peace.

And the words we'll be singing are:

Om Asato ma sat gamaya
Tamaso ma jyo tir gamaya
M'dit yir ma amritam gamaya
Om shanti shanti shanti

Translated retaining the original sentence structure we're saying: Om. From non-being me to being oh lead; from darkness me to everlasting light oh lead; from death me to eternal life oh lead. Om peace peace peace.

If you are wondering why the words aren't on paper or on the screen it is because in this practice we alternate between the oral, vocalized and aural, listened prayer practice. There is no getting it wrong. Perhaps there's no getting it right. What we are doing together is aspiring.

Walking through the meaning of the mantra part by part:

\begin{itemize}
  \item Om is the bīja representing the vibration of the universe.
  \item and then Asato ma sat gamaya: sat means being and asat means non-being and asato means from non-being; ma means me and gamaya means oh lead.
  \item on the next line Tamaso my jyo tir gamaya: tamaso is from darkness and jyo tir everlasting light, again with ma:me and gamaya:oh lead.
  \item the third part is M'dit yir ma amritam gamaya: m'dit yir is from death, ma:me; amritam is everlasting life; gamaya:oh lead.
  \item and finally Om shanti shanti shanti: Om peace peace peace.
\end{itemize}

\subsection*{Inspiring Quotations}

\begin{quote}
  The vibrations on the air are the breath of God speaking to our soul. Music is the language of God. – Ludwig van Beethoven
\end{quote}

\begin{quote}
  Where there is devotional music, God is always at hand with gracious presence. – Johann Sebastian Bach
\end{quote}

\begin{quote}
  We are as the flute, and the music in us is from thee; we are as the mountain and the echo in us is from thee. – Jalāl al-Dīn Muḥammad Rumi
\end{quote}




\subsection*{Sources}

% https://medium.com/@mmrhythm/how-to-sing-chants-vs-hymns-a-question-of-form-8b86f88fa3b3


\end{document}
