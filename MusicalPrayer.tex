\documentclass[12pt]{article}
\usepackage{lingmacros}
\usepackage{tree-dvips}
\usepackage{setspace} % for \onehalfspacing and \singlespacing macros
\onehalfspacing

\usepackage{etoolbox}
\AtBeginEnvironment{quote}{\par\singlespacing\small}

\begin{document}

\section*{Musical Prayer}

Thank you all for  the opportunity to share and further refine my understanding of musical prayer. It's been a blessing to have been raised with a great interest in and reverence for the power of music to bring us closer to the divine with dedication to the musical toolset a foundation have for the most part kept me in rewarding financial insecurity.

The seminal Hindu text The Bhagavad Gita states, “Scriptures are of little value to [one] who sees God everywhere.” For the rest of us, there are myriad systems, cosmologies, symbols, philosophies and other tools that can facilitate our journey through this mystery-filled fog of existence.

Notorious, iconoclastic nineteenth century occultist Aleister Crowley suggests that, “True religion is intoxication”, and that “all acts which excite the divine in [us] are proper to the rite of invocation.” Sufi tradition echoes this sentiment in it’s traditional spiritual metaphors of “wine” and “intoxication” or sakr (صقر) as states of divine bliss and at the very least departure from the realm of rationality. We can also achieve states of spiritual bliss through song. In yogic terms this is an aspect of the bhakti tradition: the yoga of devotion. The soul’s voicing of it’s aspiration.

Twelfth century poet, composer and mystic Hildegarde von Bingen wrote, “Don't let yourself forget that God's grace rewards not only those who never slip, but also those who bend and fall. So sing! The song of rejoicing softens hard hearts. It makes tears of godly sorrow flow from them. Singing summons the Holy Spirit. Happy praises offered in simplicity and love lead the faithful to complete harmony, without discord. Don't stop singing."

\subsection*{Loose Historical Overview}

\subsection*{Chakra System}

hymnody

Hymnologist Erik Routley lays out four traditions of hymnody in America which, “before 1900 were culturally separate, and which during the 20th century began to influence each other…”

(1) the New England Style

(2) the Southern Folk Hymnody

(3) the Black Spiritual

(4) the Gospel Song

For the congregations transitioning to Unitarianism in the 1800s, the New England style was dominant. And for this style, William Billings was the unrivaled champion. While the opening shots of this hymn battle belonged to Watts, the victory, at least in part, belongs to Billings. Billings was a prolific hymn composer and the most famous of the Singing School teachers. His musical mind was brilliant, his influence enormous, and his timing perfect. His fame and influence grew along with the rising tide of hymn singing and hymn-publishing. He published his first book of original songs in 1770. He’s credited with organizing the first church choir in America and by the late 1880s his music was known across the United States and nearly every songbook published in America had some of his songs.

In a history of the music at First Parish Concord in Massachusetts, Eleanor Billings describes the era this way..

"(Billings) was part of a whole generation of singing masters who would travel from town to town holding classes on note-reading. These evenings were especially popular among the young people in the towns, as both young men and young women were welcome, and social events were few. Those who attended the classes came to be called the ‘new way’ singers, who read notes and soon graduated to ‘anthem’ music, rather than hymn or psalm singing, and before long were seated in their own section of the church, to lead the congregational singing in lieu of deacons, and even to sing some special pieces by themselves."

\subsection*{Inspiring Quotations}

\begin{quote}
  The vibrations on the air are the breath of God speaking to our soul. Music is the language of God. – Ludwig van Beethoven
\end{quote}

\begin{quote}
  Where there is devotional music, God is always at hand with gracious presence. – Johann Sebastian Bach
\end{quote}

\begin{quote}
  We are as the flute, and the music in us is from thee; we are as the mountain and the echo in us is from thee. – Jalāl al-Dīn Muḥammad Rumi
\end{quote}




\subsection*{Sources}

https://medium.com/@mmrhythm/how-to-sing-chants-vs-hymns-a-question-of-form-8b86f88fa3b3


\end{document}
