\documentclass[12pt]{article}
\usepackage{lingmacros}
\usepackage{tree-dvips}
\usepackage{setspace} % for \onehalfspacing and \singlespacing macros
\onehalfspacing

\usepackage{etoolbox}
\AtBeginEnvironment{quote}{\par\singlespacing\small}

\begin{document}

\section*{Musical Prayer}

Thank you all for  the opportunity to share and further refine my understanding of musical prayer. It's been a blessing to have been raised with a great interest in and reverence for the power of music to bring us closer to the divine with dedication to the musical toolset a foundation have for the most part kept me in rewarding financial insecurity.

Wonderfully, this community welcomes atheists whom I hope will tolerate references to God in terms of perhaps, "the one who does not exist", "aspects of reality that defy rational interface" or some other fun applicable term.

The seminal Hindu text The Bhagavad Gita states, “Scriptures are of little value to [one] who sees God everywhere.” For the rest of us, there are myriad systems, cosmologies, symbols, philosophies and other tools that can facilitate our journey through this mystery-filled fog of existence.

Notorious iconoclastic occultist Aleister Crowley suggests that, “True religion is intoxication”, and that “all acts which excite the divine in [us] are proper to the rite of invocation.” Sufi tradition echoes this sentiment in spiritual metaphors of “wine” and “intoxication” as states of divine bliss and at the very least departure from the realm of rationality. We can also achieve states of spiritual bliss through song. In yogic terms this is an aspect of the bhakti tradition: the yoga of devotion. The soul’s voicing of its aspiration.

Luminary twelfth century poet, composer and mystic Hildegarde von Bingen wrote, “Don't let yourself forget that God's grace rewards not only those who never slip, but also those who bend and fall. So sing! The song of rejoicing softens hard hearts. It makes tears of godly sorrow flow from them. Singing summons the Holy Spirit. Happy praises offered in simplicity and love lead the faithful to complete harmony, without discord. Don't stop singing."

\subsection*{Tantra and Prana}

Two concepts that come to us from India are tantra and prana. As it was explained to me during the Vedic period between about 1500 and 200 BCE the human body was viewed as separate from and perhaps even impeding our connection with the divine. Spiritual practitioners aimed to transcend the body and the material world. In tantric philosophy beginning in the Classical period between 200 BCE and 500 CE, the body could be considered a manifestation of and map toward the divine.

This ties in with the sanscrit word, prana, which signifies breath, life and cosmic energy. Prana yama practices are techniques for cultivating and conducting energy, perception and health.

One that I'll share quickly now is kapalabhati pranayama apparently means “the pranayama that makes the forehead and entire face lustrous.” It alternates short, active exhalations with passive, reflexive inhalations. Ideally through the nose.

You can monitor the accuracy of your practice by placing a palm on your belly and holding a finger under your nose. Now inhale normally and push out a short burst of air. You are looking for the belly to contract and the finger to feel the air. You'll also hear the sound of the little air burst on the exhalation. You may at first notice the belly coming out as opposed to in on exhalation, this is a common mistake, but generally is resolved with a little practice, perhaps even resolving some of the reverse-breathing that many of us do. Oxygenating the blood and brain stimulates the digestive organs, oxygenates the blood, reduces stress reaction as well as facilitation vocalization.

\subsection*{Presentation Outline}

If I have an agenda in this presentation it's to inspire our congregation to courageously express and explore the possibilities of our engagement with music. You've been very open, welcoming and supportive musical allies and I look forward to seeing how our collective musically prayerful endeavors evolve.

Tantra, prana, science naad or nāda yoga, chakra, history

\subsection*{Loose Historical Overview}

Chanting as a spiritual practice is found in miriad cultures and religions with historical documentation going back at least as far as 1000 years before the birth of Christ. Various Old Testament books, especially the Psalms and the Chronicles, testify to the central function of music in temple worship. Psalms 150:4 which is an old testament book derived from the Torah, states, “Praise Him with the timbrel and dance; Praise Him with stringed instruments and flutes!”

Repetitive, sometimes call and response chanting is found is Byzanine, Gregorian, and other Christian traditions; Jewish cantillation, Hindu bhajan, Tibeten Guttural chant, Buddhist sutra, Sikh kirtan, prayer calls of the Muslim muezzin–the influence of which spread far and wide over 800 years of Moorish occupation in Spain; the Sufi zikr (dhikr) or calls to remember; the orisha songs of the Nigerian Yoruba; Aboriginal Australian songlines; norito chants of the Shinto religion in Japan; the chants of indigenous tribes of the Americas and many more.

Another approach to musical prayer is found in the hymn, which is a song in priase of a diety or deities. Hymns are found in the earliest of religious texts.  collection of Vedic Sanskrit hymns, the Rigveda is one of the oldest known texts in the world.

From the middle of the in ancient Egypt inscribed on the walls of the mid 14th century BCE tomb of Pharaoh Amenhotep IV was found the Great Hymn to the Aten:

\begin{quote}
How manifold it is, what thou hast made!
They are hidden from the face (of man).
O sole god, like whom there is no other!
Thou didst create the world according to thy desire,
Whilst thou wert alone: All men, cattle, and wild beasts,
Whatever is on Earth, going upon (its) feet,
And what is on high, flying with its wings
\end{quote}

Another brief passage continues:

\begin{quote}
You are in my heart,
There is no other who knows you,
\end{quote}

This hymn to the sun god Aten has been compared to Psalm 104 in the Bible, which begins "barachi nafshi et adonai", "Bless the Lord, O my soul."

Hymnologist Arthur Weigall compared the two texts side by side, commenting that "In face of this remarkable similarity one can hardly doubt that there is a direct connection between the two compositions; and it becomes necessary to ask whether both Akhnaton's hymn and this Hebrew psalm were derived from a common Syrian source, or whether Psalm 104 is derived from this Pharaoh's original poem. Both views are admissible."

Early Christians sang hymns together. A passage from New Testament Colossians reads, "Let the message of Christ dwell among you richly as you teach and admonish one another with all wisdom through psalms, hymns, and songs from the Spirit, singing to God with gratitude in your hearts". Interestingly though in Roman Catholic churches by the outset of the Protestant Reformation, "priests chanted in Latin, and choirs of professional singers predominantly sang polyphonic choral music in Latin." Musician and author Paul S. Jones writes that at this time, “there was neither congregational song nor any church music in the common tongue." It was the work of Martin Luther and John Calvin during the Protestant Reformation that brought hymn singing to the masses.

Initially hymnals didn't include music, only lyrics. The congregation would learn the tune from the choir or a song leader. In around 1800 there was bitter debate over the "old way" of Old Testament style psalm chanting and the "new way" of hymn singing.

Congregational music reading was a result of a broad social movement including regular educational hymnodic social events. A brilliant and prolific hymn composer named William Billings was the unrivaled champion of the "new way". He published his first book of original songs in 1770 and much of the text were drawn from Unitarian poets. Billings is credited with organizing the first church choir in America and by the late 1880s his music was known across the United States and nearly every songbook published in America had some of his songs.

In a history of the music at First Parish Concord in Massachusetts, Eleanor Billings describes the era this way:

"(Billings) was part of a whole generation of singing masters who would travel from town to town holding classes on note-reading. These evenings were especially popular among the young people in the towns, as both young men and young women were welcome, and social events were few. Those who attended the classes came to be called the 'new way' singers, who read notes and soon graduated to 'anthem' music, rather than hymn or psalm singing, and before long were seated in their own section of the church, to lead the congregational singing in lieu of deacons, and even to sing some special pieces by themselves."

At this time black American populations were largely illiterate and cultivated a rich call and response approach to hymn singing. Organized in 1871 out of Nashville, Tennessee's Fisk University, the Fisk Jubilee Singers toured the United States and Europe in the 1870s and 1880s, singing spirituals and hymns in the African American tradition and were credited with the popularization and preservation of this orally evolved music, a sibling of which is Gospel which combines blues and spirituals.


\subsection*{Chakra System}

hymnody, duophony
\subsection*{Inspiring Quotations}

\begin{quote}
  The vibrations on the air are the breath of God speaking to our soul. Music is the language of God. – Ludwig van Beethoven
\end{quote}

\begin{quote}
  Where there is devotional music, God is always at hand with gracious presence. – Johann Sebastian Bach
\end{quote}

\begin{quote}
  We are as the flute, and the music in us is from thee; we are as the mountain and the echo in us is from thee. – Jalāl al-Dīn Muḥammad Rumi
\end{quote}




\subsection*{Sources}

https://medium.com/@mmrhythm/how-to-sing-chants-vs-hymns-a-question-of-form-8b86f88fa3b3


\end{document}
