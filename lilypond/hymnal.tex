\documentclass[a5paper,twoside,9pt]{extbook}
\usepackage[left=.7cm,right=.7cm,top=2cm,bottom=1.5cm]{geometry}
\usepackage{Carrickc,lettrine}
\renewcommand\LettrineFontHook{\Carrickcfamily}
% http://tex.stackexchange.com/questions/250474/how-to-use-fancy-dropcaps-with-pdflatex
% http://www.tug.dk/FontCatalogue/otherfonts.html

\usepackage{graphics}
\begin{document}

\tableofcontents
%\section{}
%\subsection{}
\chapter{Introduction}

\lettrine{F}{aitheism} is the philosophy that despite the fact that God is a construct of the imagination, our spiritual convictions are nonetheless significant in navigating the vast realms of reality that are neither measurable by science nor discernible to the senses.

FUTURE

Road is a long road

Listen to Beethoven hear his soul

What is a hymn? It's a piece of music that "everyone" can sing together. It's about our place in the universe, or God, or Our relationships with existence. It probably has a small _range_ of notes so that even unaccustomed voices and approximate it, following the up and down pattern of notes across the page. According to the Oxford Dictionary:

Sounds like /him/ (somehow fitting!)

noun
1A religious song or poem, typically of praise to God or a god: a Hellenistic hymn to Apollo
More example sentences Synonyms
1.1A formal song sung during Christian worship, typically by the whole congregation.
Example sentences
1.2A song, text, or other composition praising or celebrating someone or something: a most unusual passage like a hymn to the great outdoors
More example sentences
verb
1 [with object] Praise or celebrate (something): Johnson’s reply hymns education
More example sentences
2 [no object] rare Sing hymns.

Origin

Old English, via Latin from Greek humnos 'ode or song in praise of a god or hero', used in the Septuagint to translate various Hebrew words, and hence in the New Testament and other Christian writings.

The Septuagint (from the Latin septuaginta, "seventy") is a translation of the Hebrew Bible and some related texts into Koine Greek. As the primary Greek translation of the Old Testament, it is also called the Greek Old Testament.

What is a drone?

Okay so a hymn should probably be pretty easy to sing and even read, and by that definition only a small portion of these tunes qualify, but we're lying our way to the truth.

This is a book of songs from the musical, \underline{Are We Done Yet?} by Rivka and Mike iLL of Mad haPPy.

For anyone interested in Open Source and techie stuff, this book was assembled using Lilypond music engraving package, and \LaTeX  typesetting tool via the TeXShop interface for OSX.

This is certainly a first draft, of sorts and the songs and arrangements will surely be developed as time goes on. Our own versions of most of the tracks are usually done in harmony and this first version of the book only includes one melody for each track, usually in a range for either the male or female voice, but occasionally jumping back and forth a bit.

We hope you will feel free to play with your own harmonies, re-harmonizations, new lyrics, etc. We consider this endeavor to be in the folk tradition, and some of the melodies herein are derivative of original sources.

We can be contacted, most likely via one of the following websites: \\
\begin{itemize}
\item http://www.madhappy.com 
\item http://www.rivka.com
\item http://www.mZoo.org
\item http://www.TempleofWow.com
\end{itemize}
\pagebreak{}

\section{Acknowledgements}

Is there any religion that doesn't offer some level of honor in exchange for money?
\newline
The following is a list of people who have offered significant support to this project:
  
\begin{itemize}   
  \item Georgeanna Presnell
  \item Carla Murray
  \item Marina, Hans \& Tim Frei
  \item Leah Pietrusiak
  \item Jason Daniels
\end{itemize}    

  
  Special thanks to Beth Kilmer for proof-playing the songs.
   \newline
   
   
  Cover Artwork: Seth Tobocman
  \newline
  
  Sponsored by the Puffin Foundation West.
  \includegraphics[scale=0.5]{/Users/mikekilmer/Documents/GrantWriting/PuffinFoundation/PuffinWest_2.jpg}
  \newline
  \newline
  Magical blessings to all of our wonderful friends and supporters.
  \newline
  \newline
  We hope you enjoy it.

\pagebreak{}

\section{Some Music}

Documents for \verb+lilypond-book+ may freely mix music and text.
For example,

{%
\parindent 0pt
\noindent
\ifx\preLilyPondExample \undefined
\else
  \expandafter\preLilyPondExample
\fi
\def\lilypondbook{}%
\input{ee/lily-b5562cd8-systems.tex}
\ifx\postLilyPondExample \undefined
\else
  \expandafter\postLilyPondExample
\fi
}

Options are put in brackets.

\begin{quote}
\noindent
\begin{verbatim}
  c'4 f16
\end{verbatim}
{%
\parindent 0pt
\noindent
\ifx\preLilyPondExample \undefined
\else
  \expandafter\preLilyPondExample
\fi
\def\lilypondbook{}%
\input{13/lily-cb2fc254-systems.tex}
\ifx\postLilyPondExample \undefined
\else
  \expandafter\postLilyPondExample
\fi
}
\end{quote}

So this is the Gayatri Mantra which is supposed to be super popular. We are doing the "long" version. Interestingly I read somewhere that the "short" version preceded the "long" version, historically.

Is this another paragraph?

\begin{quote}
{%
\parindent 0pt
\noindent
\ifx\preLilyPondExample \undefined
\else
  \expandafter\preLilyPondExample
\fi
\def\lilypondbook{}%
\input{1b/lily-f7e13deb-systems.tex}
\ifx\postLilyPondExample \undefined
\else
  \expandafter\postLilyPondExample
\fi
}
\end{quote}

A long road for us.

\begin{quote}
{%
\parindent 0pt
\noindent
\ifx\preLilyPondExample \undefined
\else
  \expandafter\preLilyPondExample
\fi
\def\lilypondbook{}%
\input{9b/lily-f98eabc4-systems.tex}
\ifx\postLilyPondExample \undefined
\else
  \expandafter\postLilyPondExample
\fi
}
\end{quote}

And another thing:

\begin{quote}
{%
\parindent 0pt
\noindent
\ifx\preLilyPondExample \undefined
\else
  \expandafter\preLilyPondExample
\fi
\def\lilypondbook{}%
\input{3a/lily-dcdd85c1-systems.tex}
\ifx\postLilyPondExample \undefined
\else
  \expandafter\postLilyPondExample
\fi
}
\end{quote}

(If needed, replace @file{screech-and-boink.ly} by any @file{.ly} file
you put in the same directory as this file.)

Here's a wordless melody:
\begin{quote}
{%
\parindent 0pt
\noindent
\ifx\preLilyPondExample \undefined
\else
  \expandafter\preLilyPondExample
\fi
\def\lilypondbook{}%
\input{10/lily-bd5a74ed-systems.tex}
\ifx\postLilyPondExample \undefined
\else
  \expandafter\postLilyPondExample
\fi
}
\end{quote}

And here's some more words for y'all.

\end{document}